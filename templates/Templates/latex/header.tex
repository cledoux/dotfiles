% Modeline and header {{{
% vim: set foldtext=foldtext() foldmarker={{{,}}} foldmethod=marker spell:
% }}}

% Packages {{{
\usepackage{geometry} % Used to adjust the document margins
\usepackage[square,comma,numbers,sort]{natbib} % Natbib is much more awesome for citations
\usepackage{graphicx} % Include pictures
\usepackage{grffile} % extends the file name processing of package graphics 
                     % to support a larger range 
\usepackage{listings} % Allows easy use of code in papers.
\usepackage{amssymb,amsmath,amsthm} % Equations and theorems
\usepackage{centernot} % For negating symbols
\usepackage{url}
\usepackage{longtable}
\usepackage{comment} % enable block comments
\usepackage[colorinlistoftodos]{todonotes} % enable pretty todos
\usepackage[mathletters]{ucs} % Extended unicode (utf-8) support
\usepackage[utf8x]{inputenc} % Allow utf-8 characters in the tex document
\usepackage{fancyvrb} % verbatim replacement that allows latex
\usepackage[unicode=true]{hyperref}
% Configure tables
\usepackage{booktabs} % Fancy formatting of table lines
\usepackage{longtable}
\usepackage[tableposition=top,font={small,it}]{caption}
\usepackage{tabularx}
\usepackage{tabulary}
\usepackage{algorithm}
%\usepackage{algorithmic}
\usepackage{algpseudocode}
% }}}

% New commands {{{

% Syntax cheatsheet:
%    \let\newmacro\oldmacro
%       Creates an alias between \newmacro and \oldmacro
%
%    \def \newcommand {macro expansion}
%       Creates a new macros that expands to `macro expansion`.
%       If anything even slightly complicated (such as parameters)
%       is needed, use \newcommand instead. \newcommand is a LaTex
%       wrapper around \def
%
%    \newcommand{\newcommand}[args (opt)]{command execution}
%       Create a new macro. If args is given, must be integer
%       between 0 and 9 and is number of arguments. If no args
%       omit brackets entirely. Args accessible as #1, #2, etc.

%etal stuff is not fun to type
\newcommand{\etal}{~et~al.\ } % Regular etal
\newcommand{\etalc}{~et~al.~} % Etal followed by citation
\newcommand{\etalp}{~et~al.} % Etal followed by punctuation

%Define \strong as the analog to \emph
\let\strong\textbf

% This is here mainly as documentation for the various styles of letters.
% Calligraphic 
\def \cD {\mathcal{D}}
% Blackboard
\def \bD {\mathbb{D}} 
% Fraktur
\def \fD {\mathfrak{D}}

% Various common set denotations.
% Probability
\def \P {\mathbb{P}}
% Real numbers
\def \R {\mathbb{R}}
\def \reals {\mathbb{R}}
% Counting numbers
\def \C {\mathbb{C}}

%Not implies
\newcommand{\notimplies}{\centernot\implies}

% Typeset C++
\newcommand{\cplusplus}{{\rm C\raise.5ex\hbox{\small ++}} }

% Annotation commands.
\newcommand{\added}[1]{\todo[color=green!40]{#1}}

% }}}

% New environments {{{
\newenvironment{packed_itemize}{
\begin{itemize}
  \setlength{\itemsep}{1pt}
  \setlength{\parskip}{0pt}
  \setlength{\parsep}{0pt}
}{\end{itemize}}


\newenvironment{packed_enumerate}{
\begin{enumerate}
  \setlength{\itemsep}{1pt}
  \setlength{\parskip}{0pt}
  \setlength{\parsep}{0pt}
}{\end{enumerate}}
% }}}


% Configurations {{{

% Theorem Environments {{{
% http://en.wikibooks.org/wiki/LaTeX/Theorems
\theoremstyle{definition}
\newtheorem{defenv}{Definition}
\theoremstyle{theorem}
%}}}
% listings
\lstset{
    captionpos=b,
    basicstyle=\scriptsize
}


% Automatically set image width on included graphics.
% If graphic is wider than \linewidth, will set width to \linewidth
% otherwise, just use the existing image width.
\makeatletter
\setkeys{Gin}{width=\ifdim\Gin@nat@width>\linewidth
    \linewidth
\else
    \Gin@nat@width
\fi}
\makeatother

% hyperref
% Prevent overflowing lines due to hard-to-break entities
\sloppy
\hypersetup{
    pdfborder={0 0 0},
    breaklinks=true,
    colorlinks,
    linkcolor={black},
    citecolor={black},
    filecolor={black},
    urlcolor={black},
    pdfauthor={Charles LeDoux},
    pdftitle={\thetitle},
    pdfsubject={\thedocumenttype},
    pdfkeywords={\theauthor, \thedocumenttype},
    pdfstartpage={1}
}



% url style            
\urlstyle{same}  % don't use monospace font for urls

% Margins
% \geometry{verbose,tmargin=1in,bmargin=1in,lmargin=1in,rmargin=1in}

% }}}

